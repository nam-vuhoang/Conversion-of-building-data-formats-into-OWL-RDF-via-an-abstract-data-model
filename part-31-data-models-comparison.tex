\subsection{Data model comparison}\label{sec:data-model-comparison}

All data models discussed in \autoref{sec:state-of-the-art} are driven not only by schema languages features, but also by related common practices.
\autoref{tab:data-model-comparison} shows which main features are used in EXPRESS-driven and XSD-driven building data models, and how they are supported by OWL 2 profiles.

% \blindtext
% \thispagestyle{empty}
% \begin{landscape}




\begin{table*}[t]
% \begin{sidewaystable}[t]
\footnotesize
    \centering
    \caption{Data model comparison}
    \label{tab:data-model-comparison}
    
    % \rowcolors{4}{lightgray}{}    

    \begin{threeparttable}    
        \begin{tabu} to \textwidth { X[0.5] X[0.6l] X[l] X[l] X[l] X[l] X[l] }            
            \hline            
                \multirow{2}{*}{\textbf{Features}}
                & \multicolumn{2}{c|}{\textbf{Used in building data models}}
                & \multicolumn{4}{c}{\textbf{Supported by OWL 2 profiles}}
            \\    
            \cline{2-7}            
                & \multicolumn{1}{l}{\textbf{EXPRESS-}}
                & \multirow{2}{*}{\textbf{XSD-driven}}
                & \multirow{2}{*}{\textbf{OWL 2 EL}}
                & \multirow{2}{*}{\textbf{OWL 2 QL}}
                & \multirow{2}{*}{\textbf{OWL 2 RL}}
                & \multirow{2}{*}{\textbf{OWL 2 DL}}
            \\    
                & \textbf{driven}    
                &
                &
                &
                &
                &
            \\    
            \hline            
            \hline
                \multicolumn{7}{c}{\textbf{Primitive Types}}
            \\
            % \hline            
                Decimals  % Feature  
                
                & \texttt{REAL}       % EXPRESS
                
                & \texttt{\textit{xs:decimal\tnotex{tn:rarely-used}}  \newline xs:double}      % XSD
                
                & \texttt{owl:real    \newline    owl:relational  \newline xsd:decimal}     % EL
                
                & \texttt{xsd:decimal \newline xsd:double}     % QL
                
                & \texttt{xsd:decimal \newline xsd:double}     % RL
            
                & \texttt{owl:real \newline owl:relational    \newline xsd:decimal \newline xsd:double}     % RL
            \\
            % \hline            
                Integers    % Feature
                
                & \texttt{INTEGER}       % EXPRESS
                
                & \texttt{\textit{xs:int\tnotex{tn:rarely-used}} \newline xs:integer \newline xs:long \newline
                \textit{xs:nonNegInt\tnotex{tn:xsd-nonNegInt}\tnotex{tn:rarely-used}}}  % XSD
                
                & \texttt{xsd:integer \newline xsd:nonNegInt\tnotex{tn:xsd-nonNegInt}}  % EL
                
                & \texttt{xsd:integer \newline xsd:nonNegInt\tnotex{tn:xsd-nonNegInt}}  % QL
                
                % & \texttt{xsd:int \newline xsd:integer \newline xsd:long \newline xsd:non\newline-Negative\newline-Integer}  % RL
                & \texttt{xsd:int \newline xsd:integer \newline xsd:long \newline xsd:nonNegInt\tnotex{tn:xsd-nonNegInt}}  % RL
                
                & \texttt{xsd:int \newline xsd:integer \newline xsd:long \newline xsd:nonNegInt\tnotex{tn:xsd-nonNegInt}}  % DL
            \\
            % \hline            
                Strings    % Feature
                
                & \texttt{STRING}       % EXPRESS
                
                & \texttt{\textit{xs:ID\tnotex{tn:rarely-used}  \newline xs:language\tnotex{tn:rarely-used}  \newline xs:NMTOKENS\tnotex{tn:rarely-used}}\newline
                xs:normString\tnotex{tn:normString}  \newline    xs:string}      % XSD
                
                & \texttt{xsd:normString\tnotex{tn:normString}   \newline xsd:string}    % EL
                
                & \texttt{xsd:normString\tnotex{tn:normString}   \newline xsd:string}     % QL
                
                & \texttt{xsd:normString\tnotex{tn:normString}   \newline xsd:string}     % RL
                
                & \texttt{xsd:normString\tnotex{tn:normString}   \newline xsd:string}     % DL
            \\
            % \hline            
                Booleans    % Feature
                
                & \texttt{BOOLEAN}       % EXPRESS
                
                & \texttt{xs:boolean}      % XSD
                
                & --\tnotex{no-built-in-support}    % EL
                
                & --\tnotex{no-built-in-support}     % QL
                
                & --\tnotex{no-built-in-support}     % RL
                
                & \texttt{xs:boolean}      % DL
            \\
            % \hline            
                Logicals    % Feature
                
                & \texttt{LOGICAL}       % EXPRESS
                
                & as enumeration      % XSD
                
                & --\tnotex{no-built-in-support}    % EL
                
                & --\tnotex{no-built-in-support}     % QL
                
                & --\tnotex{no-built-in-support}     % RL
                
                & --\tnotex{no-built-in-support}      % DL
            \\
            % \hline            
                Time \newline instances    % Feature
                
                & --\tnotex{no-built-in-support}       % EXPRESS
                
                & \texttt{xs:date \newline xs:dateTime \newline xs:time \newline xs:gYear}      % XSD
                
                & --\tnotex{no-built-in-support}    % EL
                
                & --\tnotex{no-built-in-support}     % QL
                
                & --\tnotex{no-built-in-support}     % RL
                
                & --\tnotex{no-built-in-support}      % DL
            \\
            % \hline            
                Names \& IDs    % Feature
                
                & --\tnotex{no-built-in-support}       % EXPRESS
                
                & \texttt{xs:anyURI \newline xs:IDREF \newline xs:QName}      % XSD
                
                & \texttt{xs:anyURI}    % EL
                
                & \texttt{xs:anyURI}     % QL
                
                & \texttt{xs:anyURI}     % RL
                
                & \texttt{xs:anyURI}      % DL
            \\
            \hline
        \end{tabu}
        \begin{tablenotes}
        %   \item\label{tn:xs-prefix} Types with prefixes \textit{xs:}
          \item\label{tn:rarely-used} Rarely used.
          \item\label{tn:a} Element declarations
          \item\label{tn:xsd-nonNegInt} \texttt{nonNegativeInteger}
          \item\label{tn:normString} \texttt{normalizedString}
          \item\label{tn:logical} Three-value logical values
          \item\label{no-built-in-support} No built-in support, requires special solutions
        \end{tablenotes}
    \end{threeparttable}    
\end{table*}
% \end{sidewaystable}

% \end{landscape}





% \subsection{Target criteria}

% \textbf{Requirement to input data models}
% ...


% \textbf{Requirement to output data formats}
% ...

% \textbf{Conversion schema}




% \subsection{Data models comparison}





% % The analysis of the simple and well-defined data models, used in the AEC/FM industry and described in Section \ref{sec:state-of-the-art}, shows that in spite of their different nature, they have a lot of similarities.

% In this subsection, the \emph{"pure"} data part of the \emph{simple} and \emph{well-defined} data models, used in the \emph{AEC/FM industry} and described in Section \ref{sec:state-of-the-art}, are compared to each other.
% "Pure" data means the static and most important information content needed for data exchange in the industry.
% Examples of "dirty" data are: worksheets and text styles in spreadsheet formats, derived attributes, functions and rules in the EXPRESS data schemas.
% "Simple" data models do not contain complex elements, such as \texttt{mixed} elements in XSD (see subsection \ref{sec:xsd-based-data-models}), \texttt{BAG}-aggregation data types in EXPRESS (see subsection \ref{sec:express-based-data-models}) or \texttt{image} in relational databases.
% "Well-defined" data models are designed by qualified researchers and engineers, using the latest computer science trends and data modeling best practices.
% The data models used outside the industry are unpredictable, for instance, their entities may not contain unique keys for linking.
% Therefore, data models, which are complicated, or "poorly" designed, or unused in the industry, are outside the scope of the research.
% In general, all the mentioned restrictions help to ensure that the data models are \emph{predictable} and probably can be mapped to the dynamic data model, specified in Section 3.2.





% In spite of different nature of the EXPRESS-based, the XSD-based, and the relational data models groups, these data models have two significant similarities:

% \begin{enumerate}

% \item
% These data models can be mapped in a certain way to an \emph{object} model. All EXPRESS-based data models are object models by definition. Most of XSD schemas are designed on top of a conceptual model, which involves two basic relationships between different "concepts": \emph{inheritance} and \emph{association}.
% The whole-part relationships (aggregations in UML), often used in XSD/XML, are special cases of associations.
% These "concepts" are reflected in certain XSD elements and can be "restored" from the XSD to be classes.
% In tabular or relational data, each table can be considered as a class, which has associations with other classes.

% \item
% Most of \emph{single data records} in datasets of these data models can be considered as entity instances.
% Examples of single data records are: single lines in SPF, rows in tabular and relational data, or XSD-elements in XML.
% However, 


% \item
% All primitive data types used in the data models are related to the following groups: strings, integers, decimals, logicals, binaries, and date-times.


% Entity types are similar to classes in OOP.
% Rows in SPF, CSV, spreadsheets and relational databases, or XSD-elements in XML are samples of single data records. 
% Rows in SPF are instances of entity types by definition.
% \end{enumerate}
% First, \emph{most} (not all) of 
% In case of tabular and relational data models, the table names can be used to define entity types.
% Second, these classes have attributes, but no operations.
% Like in EXPRESS or OWL, and unlike traditional OOP, each attribute may have multiple value.




% Third, there are only two basic relationships between these classes, namely inheritance and association.
% Each class has only one superclass. 


% ... In the Sections ..., we analyse in details each group of data models, which parts of them can or cannot be converted into OWL and RDF.


% % (4) each attribute 

% % The main difference between such a class diagram and an object model in traditional object-oriented programming (OOP) languages is that it allows attributes to have multiple values, similarly to properties in OWL languages (see...).



% % Firstly, all simple data types are related to one of the following groups: strings, integers, decimals, enumerations, calendar, and logical. 




% % The specific features of the data schemas to be ignored, for instance, are: functions, derived attributes and rules in EXPRESS; ... in XSD; data about workbooks, worksheets, text style and so on in Spreadsheets.





